\documentclass[10pt,english,a5paper]{article}
\usepackage{babel}
\usepackage[nohead,margin=.5in]{geometry}
\usepackage{parskip}
\usepackage{hyperref}
\usepackage{mathptmx}
\usepackage[scaled=.90]{helvet}
\usepackage{courier}

\renewcommand{\labelitemi}{+}

\begin{document}
\author{Dr.\ Cornelius Krasel, Matthias Andree}
\title{Leafnode FAQ}
\date{2005--10--02}
\maketitle
\tableofcontents
\section{Introduction}

This document contains answers to some commonly encountered problems.

A good way to approach communication problems with leafnode is to switch
on the debugmode. This is done by putting \verb"debugmode = 4095" into the
leafnode configuration file and then turning on the logging of the
\verb"debug" priority for the facility \verb"news" in /etc/syslogd.conf. Be warned that the leafnode programs write a lot of information to
this channel.
To turn on logging, do the following:

\begin{enumerate}

    \item Put into /etc/syslogd.conf a line which looks like the one below:
\begin{verbatim}
news.debug /var/log/news.debug
\end{verbatim}
    It is mandatory that the two fields be separated by a tab, not spaces.

\item Become root and restart the syslog daemon:
    \begin{verbatim}
kill -HUP `cat /var/run/syslog*.pid`
\end{verbatim}
    \end{enumerate}

%%%%%%%%%%

\section{Questions and Answers}

\subsection{I want to change my upstream server}

    I have never done this myself, but leafnode should have no problems
    with it. Assuming that your new server is called
    ``new.upstream.example'' and your old server
    ``old.upstream.example'', I
    recommend following the procedure outlined below:

    \begin{enumerate}
	\item Add the following lines to your config file:

	    \begin{verbatim}
server = new.upstream.example
maxfetch = 100
\end{verbatim}
       If you have already another maxfetch defined, reduce it
       temporarily.
   \item Run fetchnews
   \item Remove
       \begin{verbatim}
server = old.upstream.example
\end{verbatim}
       and delete the ``maxfetch'' line that was introduced in step 1.
       \end{enumerate}

       \subsection{I cannot connect to my newsserver}

    Most likely your setup is incorrect. This can have several reasons.

    \begin{itemize}
	\item inetd is not running. Check with
	    \begin{verbatim}
ps ax | grep inetd
\end{verbatim}
       inetd should be running once.

   \item inetd is running, but the entry in /etc/inetd.conf is incorrect.

       \begin{itemize}
	   \item The nntp entry is not recognized.
            Run \verb"inetd -d" as root (finish with Ctrl-C). You should get
            exactly one line starting like this:
	    \begin{verbatim}
ADD : nntp proto=tcp, ...
\end{verbatim}
            If this is not the case, the syntax of your entry in
            /etc/inetd.conf is incorrect. Compare it with some other
            entries. Most notably, it should not contain any leading
            spaces.

	\item The nntp entry is recognized, but leafnode is not found or
            has wrong permissions\@.

            Inetd will log attempts to invoke leafnode in some logfile,
            usually /var/log/syslog or /var/log/messages (the exact name
            depends on the setup of your /etc/syslog.conf). If an error
            occurs when invoking leafnode, you can find out there what
            exactly happened.

	\item inetd is running and the entry in /etc/inetd.conf is correct, but
       the tcpd denies access to the server (in /var/log/messages you
       will find something like ``leafnode: connection refused from\dots{}'').
       Check the entries in /etc/hosts.allow and /etc/hosts.deny. An
       example for a correct entry is given in the INSTALL file.

   \item Another badly configured news server is already running on your
       machine. Switch it off.
       \end{itemize}
       \end{itemize}

   You can test your setup by opening a telnet connection to your
   newsserver. This is done by doing \verb"telnet localhost 119". You should
   get back something like

   \begin{verbatim}
200 Leafnode NNTP Daemon, version xx running at yy
\end{verbatim}

   where xx is the version number and yy your hostname. Type \verb"quit" after
   seeing this message. If you don't get any connection at all or
   something different, check through the points above.


\subsection{tin complains about a missing file ``/var/lib/news/active''}

   Either you have started the wrong version of tin (the one which tries
   to read news directly from the spool) or your groupinfo file is
   corrupt.

   In the first case, simply invoke tin with the \verb+-r+ flag: \verb"tin -r". If
   this does not help, try to rebuild the groupinfo file by starting
   fetchnews with the \verb+-f+ flag.

   \subsection{fetchnews does not fetch any articles}

   Here are the two most common occurences which cause this error:
   \begin{itemize}
       \item Your groupinfo file is corrupt. Restart fetchnews with the -f
       parameter.
   \item /var/spool/news may have the wrong permissions. /var/spool/news
       and all its subdirectories should be drwxrwsr-x and owned by user
       and group news.
       \end{itemize}

\subsection{When fetching articles, the connection is interrupted by
the pppd}

   The article which causes the interruption contains three plus signs
   (``+++'') which is interpreted by a subset of modems as the beginning of
   a command. You can either change the command introduction sequence or
   switch off the command completely. Consult the instructions of your
   modem to find out how this is done.

\subsection{leafnode generates incorrect/incomplete message IDs}

   First you should check whether it is indeed Leafnode which generates
   the message~ID\@. Leafnode will not touch any message IDs generated by
   newsreaders (many of which generate message IDs themselves). A message
   ID that is generated by leafnode has the following general appearance:
   \verb+<local-part.ln@host-name>+

   The local-part is generated by Leafnode and you cannot influence it.
   For host-name, Leafnode tries to figure out the name of your computer
   by calling gethostname(2) and using the return value for a
   gethostbyname(3) call. Therefore, if you set the name of your computer
   correctly (using hostname(1) and domainname(1)) you should also get
   correct message IDs.

   If you don't want to change the name of your machine, you can change
   the part of the Message-ID behind the @ sign by putting
   \begin{verbatim}
hostname = my.correct.hostname.example
\end{verbatim} in your config file.
   For more information, see the leafnode(8) man page.

\subsection{fetchnews has problems retrieving new newsgroups}

   Maybe your upstream server supports neither the
   ``XGTITLE~news.group.name'' nor the
   ``LIST~NEWSGROUPS~news.group.name'' command. In
   this case, add \verb"nodesc" to your server entry as described in
   leafnode(8) and the config.example file.

\subsection{texpire does not expire articles}

   Run texpire with the -f parameter. This will expire articles somewhat
   earlier because the time of last access on the files will be ignored.
   If you have an urgent need to free some space in your spool directory,
   reduce the expiry time in the config file and re-run texpire -f.

\subsection{When running leafnode in ``delaybody'' mode, I can only
view the headers}

   This is a problem of newsreaders that cache articles, among them
   Netscape, Outlook Express GNOME Pan and KNode, not a problem of Leafnode.

   These programs store read articles in the cache and (for some odd
   reason) refuse to reload an article that has been already read.

   Try switching the cache off, clearing the cache, or use a different
   newsreader.

\subsection{I want to switch off the automatic unsubscription from
newsgroups}

   Run fetchnews with the -n parameter.

\subsection{How do I move the spool to another hard disk?}

This operation is only supported with leafnode-2.

\textbf{It is important that the hard links remain intact.}

The rest of the paragraph assumes that you wish to copy the spool from
/var/spool/news to /mnt/spool/news and that you are logged in as the
root or news user.

If you are on a GNU system, it's simple: you can just use GNU cp like this:
\begin{verbatim}
cp -ax /var/spool/news /mnt/spool/news
\end{verbatim}

If you have rsync installed, you can use (the trailing slashes ``/'' are
important):
\begin{verbatim}
rsync -aLx /var/spool/news/ /mnt/spool/news/
\end{verbatim}

If you have another cp variant than GNU, but are on a POSIX (``UNIX'')
system, you'd use instead:
\begin{verbatim}
( cd /var/spool/news && tar cf - . ) \
| (cd /mnt/spool/news && tar xpf -)
\end{verbatim}
   \end{document}
